
\section{Cahier des charges}

\subsection{Présentation du projet}

Notre projet, nommé \textit{\textbf{Blast 'em all !}}, a pour but la
surveillance de l'espace interne de la maison. Installé dans chaque pièce
de l'habitat, le système de défense est capable de s'occuper à la fois des
cibles terrestres (rats, araignées, fourmis) et des cibles aériennes
(mouches, moustiques, chauve-souris). Le système comprendra également un
module de nettoyage d'odeur et de débris afin d'effectuer sa tâche de
manière transparente pour le propriétaire de la maison. La figure n° \ref{sg}
donne un schéma général du système.


\subsection{Exigences fonctionnelles}

Nous listons ici les exigences fonctionnelles que doit remplir 
\textit{\textbf{Blast 'em all !}}.
\begin{description}
\item[Découpage de l'espace]\hfill\\
Le système s'installe pièce par pièce ; chaque pièce fonctionne ensuite de
manière indépendante. La configuration peut être globale, propre à chaque
pièce ou mixte (globale avec des exceptions).

\item[Détection]\hfill\\
Dans une pièce donnée, le système effectue un {\sl tracking} sur deux types de
cibles :
    \begin{itemize}
    \item objets au sol
    \item objets en vol
    \end{itemize}
    \vskip 6pt
Tous les objets	\textbf{en mouvement} appartenant à une des deux catégories sont
suivis. 

\item[Suppression]\hfill\\
Dans une pièce donnée, les êtres suivis sont neutralisés. La destruction 
se fait par laser pour les êtres en vol et par Tazer pour les cibles au
sol. Les espèces suivantes sont détruites par le système :
    \begin{itemize}
    \item Nuisibles volants (mouches, moustiques, papillon de nuit, mouche
    tsé-tsé, criquets, etc.)
    \item Nuisibles rampants de taille équivalente ou inférieure à celle
    d'un gros rat.
    \end{itemize}
    \vskip 6pt

\item[Purification]\hfill\\
Une fois qu'un animal a été neutralisé, des restes sont laissés au sol. Le
système doit disposer d'un module de nettoyage afin de laisser la pièce
propre après une intervention. Si plusieurs cibles sont suivis et
neutralisées, le dispositif de nettoyage doit intervenir après la dernière
élimination et optimiser son parcours de la pièce.
\end{description}

\subsection{Exigences non fonctionnelles}

Nous listons ici les exigences non fonctionnelles que doit remplir 
\textit{\textbf{Blast 'em all !}}.
\begin{description}
\item[Déontologie]\hfill\\
Le système ne devra pas intervenir sur les humains, enfants, chats, chiens, etc.
\item[Traçabilité]\hfill\\
Les différentes actions du système doivent être consignées dans un
historique de manière à assurer leur traçabilité.
\item[Ergonomie]\hfill\\
Le système doit être facilement configurable par tout type d'utilisateur.
Plusieurs interfaces de configuration doivent être prévues, plus ou moins
avancées (configuration vocale basique, par {\sl smartphone} plus avancée et via
un ordinateur experte)
\item[Évolutivité]\hfill\\
Le système doit être évolutif, aux niveaux matériel (ajout de nouvelles
pièces à surveiller) et logiciel (mise à jour)
\end{description}

\vfill
\pagebreak
