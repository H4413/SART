
\section{Conception préliminaire}

\subsection{Guide de style}
Nous avons utiliser un guide de style très simple pour le nommage des évènement et données dans les diagrammes SART:

\begin{itemize}
\item Tous les données commence par un lettre minuscule, exemple: poids.
\item Tous les évènements commence par une lettre majuscule, exemple: Appuyer.
\item Toutes les constantes commence par la lettre 'c', exemple: cTempsMax.
\end{itemize}

\subsection{Complexité du sujet}
En raison de la complexité du sujet choisie, nous avons eu besoin de mettre un peu d'inteligence aux notres bords ou faire un algorithm insufisant pour le cahier de charge. \\
Cela a été fait pour se concentrer aux objectifs du projet proposés par les enseignants.

\subsubsection{Capteur infrarouge}
Le capteur infrarouge est capable de nous envoyer des données d'une image traité, c'est-à-dire, une largeur, une hauteur, les coordonnées et la température de la cible détecté. \\
Un capteur infrarouge normal n'envoie que des longueurs d'onde qui peuvent etre traité comme une image. \\
Cependant, traité une image en utilisant SART est au-delà de la portée du projet.

\subsubsection{Déplacement du robot balayeur}
Le formalisme SART n'est pas fait pour faire de l'algorithmie lourde. C'est-à-dire, trouver le plus court chémin entre deux points et détecter tous les obstacles n'est pas de la porté de ce projet. \\
Nous avons donc choisir de faire l'algorithme le plus simple possible, c'est-à-dire, le robot se déplacera dans la dimension X jusqu'à arriver à la coordonné X de la cible et après il se déplacera sur la coordonné Y. \\
Nous supposons alors qu'il n'y a pas d'obstacle.


\vfill
\pagebreak
