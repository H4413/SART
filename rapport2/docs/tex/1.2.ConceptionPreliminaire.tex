
\section{Conception préliminaire}

\subsection{Guide de style}
Nous avons utiliser un guide de style très simple pour le nommage des évènement et données dans les diagrammes SART:

\begin{itemize}
\item Tous les données commence par un lettre minuscule,{\sl exemple: poids}.
\item Tous les évènements commence par une lettre majuscule, {\sl exemple: Appuyer}.
\item Toutes les constantes commence par la lettre  «c», {\sl exemple: cTempsMax}.
\end{itemize}

\subsection{Complexité du sujet}
En raison de la complexité du sujet choisi, nous avons eu besoin de mettre un peu d'intelligence dans nos bords
ou faire un algorithme insuffisant pour le cahier de charge. \\
Cela a été fait pour se concentrer sur les objectifs du projet proposés par les enseignants.

\subsubsection{Capteur infrarouge}
Le capteur infrarouge est capable de nous envoyer des données d'une image traitée, c'est-à-dire: une largeur,
une hauteur, les coordonnées et la température de la cible détectée. \\
Un capteur infrarouge normal n'envoie que des longueurs d'onde qui peuvent être traitées comme une image.
Cependant, traité une image en utilisant SART est au-delà de la portée du projet.

\subsubsection{Déplacement du robot balayeur}
Le formalisme SART n'est pas fait pour faire de l'algorithmique lourde. C'est-à-dire de trouver le plus court
chemin entre deux points et détecter tous les obstacles est hors de la porté de ce projet. \\
Nous avons donc choisir de faire l'algorithme le plus simple possible: le robot se
déplacera dans la dimension $X$ jusqu'à arriver à la coordonnée $X$ de la cible, puis il se déplacera sur la coordonnée $Y$. \\
Nous supposons alors qu'il n'y a pas d'obstacle.


\vfill
\pagebreak
