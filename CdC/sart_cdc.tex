\documentclass[12pt]{article}

\usepackage[T1]{fontenc}
\usepackage[utf8x]{inputenc}
\usepackage[french]{babel}

\title{\textbf{Projet SART}\\Cahier des charges}
\author{H4413\\Clément Geiger, Quentin Villers}
\date{Lundi 31 janvier 2011}

\usepackage[top=1.2in, bottom=1.2in, left=1.2in, right=1.2in]{geometry}

\begin{document}

\maketitle

\thispagestyle{empty}

\vfill
\pagebreak

\section{Présentation du projet}

Notre projet, intitulé \textit{\textbf{Blast 'em all !}}, a pour but la
surveillance de l'espace interne de la maison. Installé dans chaque pièce
de l'habitat, le système de défense est capable d'engager à la fois les
cibles terrestres (rats, araignées, fourmis) et les cibles aériennes
(mouches, moustiques, chauve-souris). Le système comprendra également un
module de nettoyage d'odeur et de débris afin d'effectuer sa tâche de
manière transparente pour le propriétaire de la maison. La figure n°1
donne un schéma général du système.


\section{Exigences fonctionnelles}

Nous listons ici les exigences fonctionnelles que doit remplir 
\textit{\textbf{Blast 'em all !}}.
\begin{description}
\item[Détection]\hfill\\
Le système s'installe pièce par pièce. Dans une pièce donnée, 
\item[Suppression]\hfill\\
Blabla
\item[Purification]\hfill\\
\end{description}

\section{Exigences non-fonctionnelles}

Nous listons ici les exigences non-fonctionnelles que doit remplir 
\textit{\textbf{Blast 'em all !}}.
\begin{description}
\item[Déontologie]\hfill\\
Le système ne devra pas intervenir sur les humains, enfants, chats, chiens, etc.
\item[Traçabilité]\hfill\\
Les différentes actions du système doivent être consignées dans un
historique de manière à assurer leur traçabilité.
\item[Ergonomie]\hfill\\
Le système doit être facilement configurable par tout type d'utilisateur.
Plusieurs interfaces de configuration doivent être prévues, plus ou moins
avancées (configuration vocale basique, par smartphone plus avancée et via
un ordinateur experte)
\item[Évolutivité]\hfill\\
Le système doit être évolutif, aux niveaux matériel (ajout de nouvelles
pièces à surveiller) et logiciel (mise-à-jour)
\end{description}

\end{document}
